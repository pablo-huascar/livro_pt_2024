% Options for packages loaded elsewhere
\PassOptionsToPackage{unicode}{hyperref}
\PassOptionsToPackage{hyphens}{url}
\PassOptionsToPackage{dvipsnames,svgnames,x11names}{xcolor}
%
\documentclass[
  letterpaper,
  DIV=11,
  numbers=noendperiod,
  oneside]{scrreprt}

\usepackage{amsmath,amssymb}
\usepackage{iftex}
\ifPDFTeX
  \usepackage[T1]{fontenc}
  \usepackage[utf8]{inputenc}
  \usepackage{textcomp} % provide euro and other symbols
\else % if luatex or xetex
  \usepackage{unicode-math}
  \defaultfontfeatures{Scale=MatchLowercase}
  \defaultfontfeatures[\rmfamily]{Ligatures=TeX,Scale=1}
\fi
\usepackage{lmodern}
\ifPDFTeX\else  
    % xetex/luatex font selection
\fi
% Use upquote if available, for straight quotes in verbatim environments
\IfFileExists{upquote.sty}{\usepackage{upquote}}{}
\IfFileExists{microtype.sty}{% use microtype if available
  \usepackage[]{microtype}
  \UseMicrotypeSet[protrusion]{basicmath} % disable protrusion for tt fonts
}{}
\usepackage{xcolor}
\usepackage[left=1in,marginparwidth=2.0666666666667in,textwidth=4.1333333333333in,marginparsep=0.3in]{geometry}
\setlength{\emergencystretch}{3em} % prevent overfull lines
\setcounter{secnumdepth}{5}
% Make \paragraph and \subparagraph free-standing
\ifx\paragraph\undefined\else
  \let\oldparagraph\paragraph
  \renewcommand{\paragraph}[1]{\oldparagraph{#1}\mbox{}}
\fi
\ifx\subparagraph\undefined\else
  \let\oldsubparagraph\subparagraph
  \renewcommand{\subparagraph}[1]{\oldsubparagraph{#1}\mbox{}}
\fi


\providecommand{\tightlist}{%
  \setlength{\itemsep}{0pt}\setlength{\parskip}{0pt}}\usepackage{longtable,booktabs,array}
\usepackage{calc} % for calculating minipage widths
% Correct order of tables after \paragraph or \subparagraph
\usepackage{etoolbox}
\makeatletter
\patchcmd\longtable{\par}{\if@noskipsec\mbox{}\fi\par}{}{}
\makeatother
% Allow footnotes in longtable head/foot
\IfFileExists{footnotehyper.sty}{\usepackage{footnotehyper}}{\usepackage{footnote}}
\makesavenoteenv{longtable}
\usepackage{graphicx}
\makeatletter
\def\maxwidth{\ifdim\Gin@nat@width>\linewidth\linewidth\else\Gin@nat@width\fi}
\def\maxheight{\ifdim\Gin@nat@height>\textheight\textheight\else\Gin@nat@height\fi}
\makeatother
% Scale images if necessary, so that they will not overflow the page
% margins by default, and it is still possible to overwrite the defaults
% using explicit options in \includegraphics[width, height, ...]{}
\setkeys{Gin}{width=\maxwidth,height=\maxheight,keepaspectratio}
% Set default figure placement to htbp
\makeatletter
\def\fps@figure{htbp}
\makeatother
\newlength{\cslhangindent}
\setlength{\cslhangindent}{1.5em}
\newlength{\csllabelwidth}
\setlength{\csllabelwidth}{3em}
\newlength{\cslentryspacingunit} % times entry-spacing
\setlength{\cslentryspacingunit}{\parskip}
\newenvironment{CSLReferences}[2] % #1 hanging-ident, #2 entry spacing
 {% don't indent paragraphs
  \setlength{\parindent}{0pt}
  % turn on hanging indent if param 1 is 1
  \ifodd #1
  \let\oldpar\par
  \def\par{\hangindent=\cslhangindent\oldpar}
  \fi
  % set entry spacing
  \setlength{\parskip}{#2\cslentryspacingunit}
 }%
 {}
\usepackage{calc}
\newcommand{\CSLBlock}[1]{#1\hfill\break}
\newcommand{\CSLLeftMargin}[1]{\parbox[t]{\csllabelwidth}{#1}}
\newcommand{\CSLRightInline}[1]{\parbox[t]{\linewidth - \csllabelwidth}{#1}\break}
\newcommand{\CSLIndent}[1]{\hspace{\cslhangindent}#1}

\KOMAoption{captions}{tableheading}
\makeatletter
\makeatother
\makeatletter
\@ifpackageloaded{bookmark}{}{\usepackage{bookmark}}
\makeatother
\makeatletter
\@ifpackageloaded{caption}{}{\usepackage{caption}}
\AtBeginDocument{%
\ifdefined\contentsname
  \renewcommand*\contentsname{Índice}
\else
  \newcommand\contentsname{Índice}
\fi
\ifdefined\listfigurename
  \renewcommand*\listfigurename{Lista de Figuras}
\else
  \newcommand\listfigurename{Lista de Figuras}
\fi
\ifdefined\listtablename
  \renewcommand*\listtablename{Lista de Tabelas}
\else
  \newcommand\listtablename{Lista de Tabelas}
\fi
\ifdefined\figurename
  \renewcommand*\figurename{Figura}
\else
  \newcommand\figurename{Figura}
\fi
\ifdefined\tablename
  \renewcommand*\tablename{Tabela}
\else
  \newcommand\tablename{Tabela}
\fi
}
\@ifpackageloaded{float}{}{\usepackage{float}}
\floatstyle{ruled}
\@ifundefined{c@chapter}{\newfloat{codelisting}{h}{lop}}{\newfloat{codelisting}{h}{lop}[chapter]}
\floatname{codelisting}{Listagem}
\newcommand*\listoflistings{\listof{codelisting}{Lista de Listagens}}
\makeatother
\makeatletter
\@ifpackageloaded{caption}{}{\usepackage{caption}}
\@ifpackageloaded{subcaption}{}{\usepackage{subcaption}}
\makeatother
\makeatletter
\@ifpackageloaded{tcolorbox}{}{\usepackage[skins,breakable]{tcolorbox}}
\makeatother
\makeatletter
\@ifundefined{shadecolor}{\definecolor{shadecolor}{rgb}{.97, .97, .97}}
\makeatother
\makeatletter
\makeatother
\makeatletter
\@ifpackageloaded{sidenotes}{}{\usepackage{sidenotes}}
\@ifpackageloaded{marginnote}{}{\usepackage{marginnote}}
\makeatother
\makeatletter
\makeatother
\ifLuaTeX
\usepackage[bidi=basic]{babel}
\else
\usepackage[bidi=default]{babel}
\fi
\babelprovide[main,import]{brazilian}
% get rid of language-specific shorthands (see #6817):
\let\LanguageShortHands\languageshorthands
\def\languageshorthands#1{}
\ifLuaTeX
  \usepackage{selnolig}  % disable illegal ligatures
\fi
\IfFileExists{bookmark.sty}{\usepackage{bookmark}}{\usepackage{hyperref}}
\IfFileExists{xurl.sty}{\usepackage{xurl}}{} % add URL line breaks if available
\urlstyle{same} % disable monospaced font for URLs
\hypersetup{
  pdftitle={Produções Técnicas em Psicologia},
  pdfauthor={Francisco Pablo Huacar Aragão Pinheiro; João Paulo Pereira Barros},
  pdflang={pt-br},
  colorlinks=true,
  linkcolor={blue},
  filecolor={Maroon},
  citecolor={Blue},
  urlcolor={Blue},
  pdfcreator={LaTeX via pandoc}}

\title{Produções Técnicas em Psicologia}
\usepackage{etoolbox}
\makeatletter
\providecommand{\subtitle}[1]{% add subtitle to \maketitle
  \apptocmd{\@title}{\par {\large #1 \par}}{}{}
}
\makeatother
\subtitle{Experiências Compartilhadas}
\author{Francisco Pablo Huacar Aragão Pinheiro \and João Paulo Pereira
Barros}
\date{}

\begin{document}
\maketitle
\ifdefined\Shaded\renewenvironment{Shaded}{\begin{tcolorbox}[sharp corners, borderline west={3pt}{0pt}{shadecolor}, interior hidden, enhanced, frame hidden, boxrule=0pt, breakable]}{\end{tcolorbox}}\fi

\renewcommand*\contentsname{Sumário}
{
\hypersetup{linkcolor=}
\setcounter{tocdepth}{2}
\tableofcontents
}
\bookmarksetup{startatroot}

\hypertarget{apresentauxe7uxe3o}{%
\chapter*{Apresentação}\label{apresentauxe7uxe3o}}
\addcontentsline{toc}{chapter}{Apresentação}

\markboth{Apresentação}{Apresentação}

Lorem ipsum dolor sit amet, consectetur adipiscing elit. Sed in sagittis
ipsum. Ut et semper ligula, pretium sollicitudin orci. Ut in est ac
dolor finibus volutpat. Mauris mattis lacus mi, quis ullamcorper nulla
tempus vel. Integer quis est vitae nunc lobortis ultrices. Proin
efficitur bibendum turpis vel iaculis. Pellentesque tristique viverra
elit, eu dignissim dui ornare ac. Suspendisse tellus lectus, fermentum
vel vehicula ut, efficitur non neque. Aliquam leo leo, congue a cursus
sed, eleifend lacinia massa. Duis ex ligula, porttitor eu tristique ac,
mattis vitae lacus. Donec condimentum libero metus, maximus tempus massa
tincidunt et. Maecenas rutrum dolor ut quam tempus sagittis. Aliquam
nisl lorem, sollicitudin at sem sed, posuere lobortis dui. Pellentesque
vel arcu turpis. Donec fringilla lacus condimentum risus efficitur, eu
tempus massa volutpat.

Etiam libero ipsum, facilisis ac erat nec, sagittis vulputate est.
Quisque ac pellentesque lorem. Nullam nec massa sed enim facilisis
porttitor at sit amet metus. Maecenas nulla enim, dictum ut orci ac,
auctor vulputate metus. Nulla facilisi. Suspendisse sollicitudin, nisi
eu vulputate tempus, urna ligula ornare urna, sit amet malesuada arcu
velit at nisl. Vestibulum tincidunt vel dolor ut condimentum. Donec
sagittis euismod fringilla. Etiam consequat vitae erat sit amet
pellentesque. Nunc sapien nisl, malesuada eget mi non, pretium
sollicitudin nulla.

Nullam commodo mauris urna, vel accumsan dolor pellentesque sit amet.
Aenean maximus et libero nec cursus. Nullam ut ultricies velit, id
ultrices mi. Nulla ut felis malesuada, hendrerit velit ut, malesuada
tortor. Quisque finibus purus et orci sagittis, eget pulvinar turpis
tincidunt. Nam dapibus, nisi in lobortis elementum, purus est porta
neque, nec suscipit urna urna at urna. Etiam at pulvinar mauris.

Nunc libero elit, ultricies sed feugiat sit amet, venenatis id felis.
Vivamus nibh justo, ultricies sed laoreet eget, faucibus sit amet massa.
Aenean volutpat varius massa. Vestibulum viverra diam quis enim bibendum
pharetra. Vivamus non sodales neque. Aenean lobortis ligula sed mi
ultricies eleifend. Nulla congue commodo vulputate. Vivamus tincidunt
facilisis quam, eu maximus erat vehicula eget. Ut porttitor, turpis ac
dapibus fermentum, risus est lacinia tortor, imperdiet mattis enim orci
eu metus. Fusce mollis malesuada lacus, consectetur mollis velit finibus
non. Orci varius natoque penatibus et magnis dis parturient montes,
nascetur ridiculus mus. Integer condimentum rhoncus arcu, in commodo ex
sollicitudin id. Curabitur tincidunt eget magna a pulvinar. Nam
consequat mollis cursus. Sed accumsan lacus nisl, sit amet egestas lacus
elementum id.

Donec ultricies nisi in ex facilisis porta. Nam vitae diam rhoncus,
imperdiet est eget, pretium neque. Fusce euismod, justo nec feugiat
dignissim, ex arcu malesuada urna, quis elementum lorem purus ut est.
Aliquam laoreet vel purus id lacinia. Vestibulum accumsan justo vitae
nisi imperdiet ultrices. Nulla ornare sapien id lacus dignissim viverra.
Phasellus fringilla auctor turpis, ut hendrerit dolor faucibus vel.

\bookmarksetup{startatroot}

\hypertarget{expediente-institucional}{%
\chapter*{Expediente institucional}\label{expediente-institucional}}
\addcontentsline{toc}{chapter}{Expediente institucional}

\markboth{Expediente institucional}{Expediente institucional}

\textbf{Presidente da república}

Luiz Inácio Lula da Silva

\textbf{Ministro da educação}

Camilo Sobreira de Santana

\textbf{Universidade Federal do Ceará -- UFC}

\marginnote{\begin{footnotesize}

\begin{figure}

{\centering 

\href{https://www.ufc.br/}{\includegraphics[width=4.375in,height=\textheight]{imagens/brasao1_horizontal_cor_300dpi.png}}

}

\end{figure}

\end{footnotesize}}

\textbf{Reitor}

Prof.~Custódio Luís Silva de Almeida

\textbf{Vice-reitora}

Profa Diana Cristina Silva de Almeida

\textbf{\emph{Programa de pós-graduação em Psicologia}}

\textbf{Coordenador}

Professor Aluísio Ferreira de Lima

\textbf{Vice-coordenador}

Professor Paulo Coelho Castelo Branco

\textbf{\emph{Programa de pós-graduação profissional em Psicologia e
políticas públicas}}

\marginnote{\begin{footnotesize}

\begin{figure}

{\centering 

\href{https://psipolpublicas.ufc.br/pt/}{\includegraphics[width=4.375in,height=\textheight]{imagens/logo_pos.png}}

}

\end{figure}

\end{footnotesize}}

\textbf{Coordenador}

Professor Francisco Pablo Huascar Aragão Pinheiro

\textbf{Vice-coordenador}

Professor Paulo Henrique Dias Quinderé

\bookmarksetup{startatroot}

\hypertarget{folha-de-rosto}{%
\chapter*{Folha de rosto}\label{folha-de-rosto}}
\addcontentsline{toc}{chapter}{Folha de rosto}

\markboth{Folha de rosto}{Folha de rosto}

\textbf{Produções Técnicas em Psicologia: Experiências Compartilhadas}

Este livro é distribuído sob os termos da
\href{https://creativecommons.org/licenses/by-nc-nd/4.0/}{CC BY-NC-ND
4.0 DEED}.

\textbf{Publicado no Brasil}

\emph{Programa de pós-graduação em Psicologia - Universidade Federal do
Ceará}

Av. da Universidade, 2762, Benfica - CEP: 60020-180 - Fortaleza/CE --
Área 2 do Centro de Humanidades - Bloco Didático Prof.~Ícaro de Sousa
Moreira - 1º andar.

\emph{Programa de pós-graduação profissional em Psicologia e políticas
públicas - Universidade Federal do Ceará}

Rua Coronel Estanislau Frota, 563 -- Centro -- CEP 62010-560 -- Sobral
-- CE -- Campus Sobral -- Mucambinho

\textbf{Conselho editorial}

Cassio Adriano Braz de Aquino - UFC

Elisa Lacerda-Vandenborn - Unversity of Calgary (Canadá)

Jorge Tarcísio da Rocha Falcão - UFRN

Lara Brum de Calais - UFES

Luis Achilles Rodrigues Furtado - UFC

\textbf{Revisão de texto}

João Paulo Silva Matos

(joaopaulosmatos2@gmail.com)

\textbf{Codificação para construção do livro}

Francisco Pablo Huascar Aragão Pinheiro

\bookmarksetup{startatroot}

\hypertarget{references}{%
\chapter*{References}\label{references}}
\addcontentsline{toc}{chapter}{References}

\markboth{References}{References}

\hypertarget{refs}{}
\begin{CSLReferences}{0}{0}
\end{CSLReferences}

\part{Produtos de Comunicação}

\part{Eventos Organizados}

\part{Relatórios Técnicos - Assessoria, Consultoria e Pesquisa}

\part{Curso para Formação Profissional}

\part{Material Didático}

\part{Compilações de Vários Produtos Técnicos}

\part{Manual/Protocolo}

\hypertarget{processo-de-criauxe7uxe3o-do-miniguia-do-orientador-educacional-sobre-suas-atribuiuxe7uxf5es-e-indicadores-de-efeitos-no-contexto-escolar-sobralense}{%
\chapter{Processo de criação do miniguia do orientador educacional sobre
suas atribuições e indicadores de efeitos no contexto escolar
sobralense}\label{processo-de-criauxe7uxe3o-do-miniguia-do-orientador-educacional-sobre-suas-atribuiuxe7uxf5es-e-indicadores-de-efeitos-no-contexto-escolar-sobralense}}

\hypertarget{introduuxe7uxe3o}{%
\section{Introdução}\label{introduuxe7uxe3o}}

O capítulo trata da apresentação do processo de criação do Miniguia do
Orientador Educacional: Atribuições e Indicadores de Efeitos no Contexto
Escolar Sobralense, na qualidade de produto técnico desenvolvido no
Mestrado Profissional em Psicologia e Políticas Públicas, entre os anos
de 2020 e 2022. Ele é originário de prévia pesquisa empírica e
documental submetida como artigo científico com o título
``Caracterização do Perfil Profissional e Modos de Atuação do Orientador
Educacional na Rede Pública de Ensino de Sobral-CE'' e apresenta o
resultado da análise das atribuições do orientador educacional (OE)
descritas na literatura nacional, nas normativas da profissão e na
experiência relatada pelos OEs de Sobral. Com isso, foi possível
sistematizar atribuições e, a partir dessas, desenvolver indicadores de
efeitos e sugestões de ações aplicáveis no contexto descrito, as quais
foram organizadas no Miniguia.

No Brasil, a profissão de orientador educacional foi criada em 1968 (Lei
nº 5.564) e regulamentada em 1973 (Decreto-lei nº 72.846). Até então, o
cargo de orientador educacional era destinado a licenciados em pedagogia
e/ou profissionais que já fizessem parte do corpo escolar, sendo
remanejados para esse cargo de acordo com a necessidade da gestão
escolar ou das secretarias municipais (PASCOAL, 2008).

Em Sobral, graduados em psicologia adentraram às escolas com o cargo de
orientador educacional por meio da criada carreira de apoio à gestão
escolar (Lei nº 1704/2017), que oportunizou concurso público com 50
vagas para o referido cargo. Esse certame tinha por finalidade contratar
profissionais para dar prosseguimento à política pública de
desenvolvimento de competências socioemocionais nas escolas municipais
dos anos iniciais e finais do Ensino Fundamental (1º a 9º ano).

Com a inserção da primeira autora dentre os OEs admitidos nas escolas de
Sobral em 2019 e com sua aprovação no Mestrado Profissional de
Psicologia e Políticas Públicas da Universidade Federal do Ceará, sob
orientação da segunda autora, sua pesquisa de mestrado teve como
finalidade traçar o perfil profissional do OE e esquadrinhar a relação
deste, no contexto sobralense, com o que está descrito na literatura
nacional e em documentos oficiais.

Os dados aqui apresentados para a elaboração do Miniguia têm como base
as informações coletadas de 11 orientadores educacionais, convocados na
primeira chamada do concurso, dos quais 9 são mulheres, e 2 homens,
todos entre 24 e 36 anos. Desse conjunto, 2 atuam em escolas com ambos
os níveis de ensino (anos iniciais e finais), 2 em escolas especialistas
de tempo integral de anos finais, 3 em escolas dos dois níveis de ensino
(regular e integral), 2 exclusivamente em escolas de tempo integral, 1
em escolas exclusivamente de anos finais e iniciais, 02 somente em
escola regular de anos finais, tanto em áreas urbanas quanto rurais.

Dito isto, descreveremos a estruturação do Miniguia do Orientador
Educacional: Atribuições e Indicadores de Efeitos no Contexto Escolar
Sobralense, seguindo as seções: Apresentação; Ser orientador
educacional; O que são atribuições profissionais?; Tipos de atribuições
e possíveis indicadores de efeitos; Objetivos das atribuições;
Caracterização das ações por tipo de atribuições; e Dicas.

\#\# Apresentação

Esta seção pretende apresentar ao leitor uma visão geral da pesquisa que
resultou no Miniguia. Explicita-se aqui que o material construído não
representa a totalidade da caracterização profissional do orientador
educacional, e sim um recorte que pode ser considerado como um ponto de
partida para aqueles que já estão no cargo ou para os que futuramente
possam vir a estar.

\hypertarget{ser-orientador-educacional}{%
\section{Ser orientador educacional}\label{ser-orientador-educacional}}

A seção ``Ser Orientador Educacional'' traz uma descrição do Orientador
Educacional no contexto municipal como aquele que tem formação acadêmica
em Psicologia e dispensa necessidade de qualificação profissional para a
execução de suas funções, tais como pós-graduação. Apesar de não haver
exigência de qualificação, observamos que, no grupo dos 11 orientadores
educacionais, há um maior percentual de qualificação na área da
educação, concentrando 49,9\% dos integrantes, sendo 14,3\% deles com
especialização, 21,4\% com mestrado em andamento, 7,1\% com mestrado
concluído e 7,1\% com especialização em andamento.

Em Sobral, a formação dos OEs e o acompanhamento de suas ações foram
levados a cabo pela parceria entre o Instituto Ayrton Senna, a
Vice-Governadoria do Estado e a Escola de Formação Permanente de
Magistério e Gestão Educacional (ESFAPEGE). De acordo com as orientações
dessas instituições, o trabalho do OE envolve todos os atores escolares,
priorizando o desenvolvimento integral do aluno. Além disso,
caracteriza-se aquele como o profissional que promove a prática de
acolhimento e escuta, provocando os demais atores escolares a
desenvolver a comunicação não-violenta e dar espaço para as expressões
emocionais, de modo a promover o desenvolvimento integral dos sujeitos.

\hypertarget{o-que-suxe3o-atribuiuxe7uxf5es-profissionais}{%
\section{O que são Atribuições
Profissionais?}\label{o-que-suxe3o-atribuiuxe7uxf5es-profissionais}}

No tópico ``Atribuições Profissionais'', o Miniguia apresenta as
atribuições do OE e as sistematiza em duas categorias: atribuições
privativas e atribuições complementares. No volume, também são
apresentadas as principais normativas sobre as atribuições dos OEs,
tanto em âmbito local quanto em nacional. Nesse ponto do guia, as ações
dos OEs foram agrupadas de modo a distingui-las de atividades
desempenhadas por outras categorias profissionais. Além disso, as
atribuições específicas dos OE de Sobral foram divididas em: primárias e
secundárias.

As atribuições categorizadas como primárias reuniram aquelas atividades
consideradas pelos participantes da pesquisa como demandas em cuja
execução estes são figuras centrais (o que diferencia sua ação das dos
demais atores escolares). As atribuições secundárias para os OE são
aquelas centradas em outros atores escolares, sendo o orientador
educacional um apoio para a execução destas. As primárias são compostas
por dois eixos, quais sejam Eixo 1 -- Construção de Vínculos e Eixo 2 --
Práticas Pedagógicas. As secundárias, por sua vez, envolvem o orientador
educacional como participante em vez de executor da ação. Em face disso,
o Miniguia também apresenta o fluxograma dessas atribuições por eixo e
público-alvo.

\hypertarget{tipos-de-atribuiuxe7uxf5es-e-possuxedveis-indicadores-de-efeitos}{%
\section{Tipos de atribuições e Possíveis Indicadores de
efeitos}\label{tipos-de-atribuiuxe7uxf5es-e-possuxedveis-indicadores-de-efeitos}}

Os indicadores de efeitos das intervenções foram propostos com base na
especificação das funções de cada atribuição, as quais foram organizadas
nas seguintes categorias: Instrução, Inserção e Interação. Ainda, os
indicadores de efeitos das atividades foram sistematizados de modo a
trazer mais clareza sobre como avaliar o trabalho do OE e a propor
mudanças baseadas em dados mensuráveis.

Através dos dados analisados na pesquisa, a seção ``Tipos de
atribuições'' traz 4 tipos de atribuições, dentro das primárias e
secundárias, nas quais foi possível sugerir indicadores de efeitos,
sendo elas: 1. Formações; 2. Mediação/Ações de cuidado/Ações coletivas;
3. Reuniões/Suporte/Participação; 4.
Acompanhamento/Elaboração/Observação/Planejamento. Cada um desses grupos
de tipos de atribuições tem seus indicadores de efeito descritos no
Miniguia de forma detalhada.

\hypertarget{objetivos-da-atribuiuxe7uxf5es}{%
\section{Objetivos da
atribuições}\label{objetivos-da-atribuiuxe7uxf5es}}

Compreender as atribuições a partir dos seus objetivos tem como função
auxiliar o orientador educacional a conhecer o modo como essas
atribuições são planejadas e a forma como o resultado de sua atuação é
mensurado, por meio dos indicadores de efeitos oferecidos pelo Miniguia
ou criados pelo próprio OE.

\hypertarget{caracterizauxe7uxe3o-das-auxe7uxf5es-por-tipo-de-atribuiuxe7uxe3o}{%
\section{Caracterização das ações por tipo de
atribuição}\label{caracterizauxe7uxe3o-das-auxe7uxf5es-por-tipo-de-atribuiuxe7uxe3o}}

Depois de tudo pronto, faltava responder a uma das maiores inquietações
dos orientadores educacionais: como priorizar o público-alvo quando tudo
parece ser tão urgente?

Para isso, foi criada a seção ``Caracterização de ações por tipo de
atribuição'', com sugestões de ações-padrão para determinado tipo de
atribuição.

Como um exemplo disso observamos que, se o tipo de atribuição é
formação, o orientador pode ter como sugestão de metodologia as oficinas
e os projetos. De outro modo, em se tratando de alguma atribuição de
mediação, ele pode voltar-se para rodas temáticas, círculo de diálogo ou
resolução de conflitos.

Com esses exemplos, procuramos, nessa seção, trazer possibilidades de
metodologias de trabalho para cujo exercício eles foram qualificados, às
quais, contudo, o OE não deve se limitar a elas. Isso porque aqui se
trata mais de uma questão de já haver materiais de base, ofertados pelos
parceiros da SEDUC, que se caracterizam como ações esperadas e demandas
prioritárias no trabalho do Orientador Educacional.

\hypertarget{dicas}{%
\section{Dicas}\label{dicas}}

As dicas são uma forma de visualizar estratégias de intervenção para as
demandas do orientador educacional, buscando sistematizar ações para a
questão ``Por onde devo começar?'', para a identificação do perfil da
atividade e de alguns pontos de atenção.

\hypertarget{considerauxe7uxf5es}{%
\section{Considerações}\label{considerauxe7uxf5es}}

O trabalho do orientador educacional (OE) no contexto educacional
sobralense ainda está em processo de construção, moldando-se conforme as
demandas e ajustando-se às atribuições em função disso. Esse fato foi
considerado quando nos propusemos a estudar o perfil e os modos de
atuação desse profissional para elaborar o Miniguia.

Esses profissionais trazem um campo novo de atuação para as escolas,
enfrentam resistências e desafios em suas jornadas uma vez que, até
então, não havia caminhos já traçados ou formas de fazê-lo. Todo o
processo, dessa forma, vem sendo construído pelos profissionais
referidos. Além disso, apesar de terem a mesma formação acadêmica e
passarem pelas mesmas qualificações, esses profissionais atuam em
territórios diversos, o que exige do profissional um manejo e uma
adaptação constantes à realidade vivenciada. A isso se soma, ainda, o
fato de nem sempre essas estratégias adaptativas serem repassadas nas
formações às quais o orientador deve estar presente.

Diante do apresentado, acreditamos que ainda há muito a ser explorado no
que tange às atribuições e à criação de indicadores de efeitos, assim
como estratégias que busquem superar os obstáculos -- os quais essa
pesquisa buscou sistematizar no intento de, então, apresentar-lhes
possibilidades -- identificados pelos orientadores.



\end{document}
